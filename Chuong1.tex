\chapter{TỔNG QUAN}
\section{Cơ bản về hệ thống quản lý phiên bản (VCS)}
\hspace{1cm}Trong thời đại ngày nay, công nghệ thông tin phát triển mạnh mẽ mang đến nhiều ứng dụng đa dạng trong đời sống. Máy tính, điện thoại thông minh trở thành thiết bị phổ biến, phục vụ cho cả học tập, làm việc và giải trí. Việc xem phim là nhu cầu giải trí của nhiều người, và website xem phim trực tuyến ra đời đáp ứng nhu cầu đó một cách tiện lợi và hiệu quả.
Phim là một hình thức nghệ thuật đa phương tiện mà thông qua việc sử dụng hình ảnh, âm thanh và diễn xuất, kể một câu chuyện hoặc truyền đạt một ý tưởng đến khán giả. Phim có thể được sản xuất dưới nhiều dạng khác nhau như phim điện ảnh, phim truyền hình, hoặc video giải trí.
Website xem phim là một nền tảng trực tuyến cung cấp dịch vụ xem phim trực tuyến đa dạng và phong phú. Với sứ mệnh mang lại niềm vui và sự thư giãn cho người xem, chúng tôi cung cấp một thư viện phim đa thể loại từ hành động, hài hước, tình cảm đến khoa học viễn tưởng. Người dùng có thể truy cập và xem phim mọi lúc, mọi nơi từ các thiết bị di động hay máy tính cá nhân chỉ cần có kết nối Internet. Điều này tạo ra sự tiện lợi và linh hoạt cho người xem. Website xem phim không chỉ là nơi xem phim mà còn là cộng đồng trực tuyến cho những người đam mê điện ảnh. Người dùng có thể chia sẻ cảm nhận, đánh giá phim, và thảo luận với nhau về các tác phẩm điện ảnh.

\section{PHẠM VI NGHIÊN CỨU}
\hspace{1cm}Để giới hạn phạm vi nghiên cứu, báo cáo tập chung xem xét, phân tích, đánh giá các yếu tố nằm trong khu vực.

\begin{itemize}
    \item Địa điểm nghiên cứu: TP Cần thơ.
    \item Hoạt động nghiên cứu: tập trung vào việc thiết kế, phát triển, và thử nghiệm các tính năng và chức năng của trang web. Nghiên cứu cũng bao gồm việc tìm hiểu về thị trường trực tuyến xem phim, thu thập và phân tích dữ liệu về sở thích của người dùng, cũng như đánh giá hiệu suất và sự hài lòng của họ.
    \item Thời gian: 5-6 tháng.
\end{itemize}
\section{MỤC TIÊU VÀ PHƯƠNG PHÁP NGHIÊN CỨU}
\subsection{Mục tiêu}
Tạo ra 1 website xem phim ở ba mức độ: người dùng, biên tập viên và quản trị.
\begin{itemize}
    \item Nội dung của website người dùng:
    \begin{itemize}
        \item Trang chủ.
        \item Xem thông tin phim.
        \item Xem phim.
        \item Thông tin người dùng.
        \item Đổi mật khẩu.
        \item Xem danh sách phim yêu thích.
        \item Thông báo.
        \item Gửi ý kiến phản hồi.
        \item Tìm kiếm.
    \end{itemize}
     \item Nội dung của website biên tập viên:
    \begin{itemize}
        \item Tổng quan.
        \item Quản lý phim.
    \end{itemize}
    \item Nội dung của website nhà quản trị:
    \begin{itemize}
        \item Tổng quan.
        \item Quản lí người dùng.
        \item Quản lí báo cáo.
        \item Quản lí bình luận.
        \item Quản lí phim.
    \end{itemize}
\end{itemize}
\subsection{Phương pháp nghiên cứu}
\begin{itemize}
    \item Bối cảnh nghiên cứu:
Bối cảnh nghiên cứu của web xem phim bao gồm nhu cầu thị trường ngày càng cao, sự phát triển của công nghệ và một số vấn đề cần giải quyết. Việc nghiên cứu bối cảnh này sẽ giúp các nhà phát triển web xem phim hiểu rõ hơn về thị trường, đối thủ cạnh tranh và nhu cầu của người dùng, từ đó tạo ra một website xem phim chất lượng cao và đáp ứng nhu cầu của người dùng.
    \item Tổng thể nghiên cứu và chọn mẫu
    
Trong quá trình nghiên cứu và chọn mẫu của trang web xem phim, việc xác định mục tiêu nghiên cứu, phân tích yêu cầu và mong muốn của người dùng, cũng như lựa chọn phương pháp nghiên cứu phù hợp là các bước quan trọng. Đồng thời, quá trình chọn mẫu cũng đòi hỏi sự cân nhắc kỹ lưỡng để đảm bảo mẫu được lựa chọn đại diện cho đối tượng nghiên cứu một cách chính xác và đáng tin cậy.
    \item Phương pháp thu thập số liệu

Trong phương pháp nghiên cứu về trang web xem phim, phương pháp thu thập số liệu thường bao gồm việc sử dụng các công cụ phân tích web để thu thập dữ liệu về lượt truy cập, thời lượng xem phim, tương tác người dùng, và các thông tin khác liên quan đến hoạt động trên trang web. Ngoài ra, việc tiến hành khảo sát trực tuyến, phỏng vấn trực tuyến, hoặc phân tích dữ liệu từ cơ sở dữ liệu có thể được áp dụng để thu thập thông tin đa chiều và đa dạng cho nghiên cứu.
    \item Phương pháp xử lí thông tin

Dữ liệu thu thập được từ báo cáo và khảo sát đã được xử lí và phân tích bằng sử dụng phương pháp phân tích số liệu thống kê. Chúng tôi đã sử dụng các công cụ và kỹ thuật phân tích biến thể, phân tích mục tiêu, và phân tích phương sai để khám phá các mẫu, xu hướng và mối quan hệ trong dữ liệu. Điều này giúp chúng tôi hiểu rõ hơn về hành vi và sở thích của người dùng đọc truyện trên web.
    \item Xây dựng mô hình

Dựa trên việc phân tích dữ liệu thu thập được từ các công cụ phân tích web, khảo sát trực tuyến, hoặc phỏng vấn người dùng. Quá trình xử lý thông tin có thể bao gồm việc sắp xếp, phân loại, và phân tích dữ liệu để đưa ra những kết luận và nhận định có ý nghĩa về hoạt động và hiệu suất của trang web xem phim. Các phương pháp thống kê và phân tích dữ liệu cũng được áp dụng để đảm bảo tính chính xác và đáng tin cậy của kết quả nghiên cứu.
\end{itemize}

\section{BỐ CỤC}
Đồ án gồm có 5 chương:
\begin{itemize}
    \item Chương 1: Tổng quan gồm: lý do chọn đề tài, mục tiêu và phương pháp nghiên cứu, phạm vi nghiên cứu, bố cục.
    \item Chương 2: Cơ sở lý thuyết bao gồm:
        ducanh2912/next-pwa,
        emotion/react,
        emotion/styled,
        mui/icons-material,
        mui/lab,
        mui/material,
        mui/x-charts,
        mui/x-data-grid,
        mui/x-date-pickers,
        reduxjs/toolkit,
        tippyjs/react,
        trendyol-js/react-carousel,
        bcrypt,
        bootstrap,
        classnames,
        dayjs,
        hls.js,
        mongodb-cloudflare,
        next,
        next-auth,
        next-client-cookies,
        normalize.css,
        react-bootstrap,
        react-easy-crop,
        react-player,
        react-redux,
        react-select,
        react-slick,
        screenfull,
        slick-carousel,
        react,
        react-dom,
        cloudflare/next-on-pages,
        cloudflare/workers-types,
        mui/x-data-grid-generator,
        types/classnames,
        types/node,
        types/react,
        types/react-dom,
        types/react-slick,
        sass,
        typescript,
        vercel,
        webpack,
        wrangler.
    
    \item Chương 3: Phân tích thiết kế hệ thống gồm sơ đồ phân cấp chức năng, sơ đồ thực thể quan hệ, mô hình cơ sở dữ liệu.
    \item Chương 4: Xây dựng hệ thống các giao diện website.
    \item Chương 5: Kết quả thực hiện: Kết quả đạt được, hạn chế và hướng phát triển.
\end{itemize}
